\chapter{Introduction}

In a typical corporate environment there exists multiple conference/meeting rooms.
The site that we studied was the  Fractal Analytic, Goregaon. The office has around 400 employees: 250 employees on 7th floor, 150 employees on 3rd floor. 7th floor has 10 meeting rooms and 3rd floor has 5 meeting rooms. Anyone can book any meeting room for any time (if the room is available) using a mobile app. This is an open office - hence if anyone wants to have a discussion then they need to go to a meeting room. Hence meeting rooms are always in demand.

\vspace{20pt}

The problem was that anyone could book a meeting room and then not use it. Or if someone wanted to have a meeting without prebooking the meeting room the he/she would have to go from room to room to check the availability of the rooms. The would create a lot of unnecessary hassle and would lead to unoptitmal utilization of the workspace.

\vspace{20pt}

To solve these problem we envisioned a solution which would involve placing sensors in every meeting room to monitor the occupancy of the room. This sensor will then relay the real time occupancy information of the room to a central device which will keep track of the occupancy status of all the meeting rooms. This data can then be showed on a web/mobile interface to check the real time occupancy status of the meeting room or the data can be integrated with the the mobile app to automatically book or cancel the meetings as per the occupancy of the rooms.